\documentclass[14pt,xcolor=svgnames]{beamer}
\usepackage{mesImports}
%\usepackage{diapos}
\usepackage{mathapp}
\usepackage{multimedia}


\title[Skeleton animation]
{Skeleton animation - bridging Maya and OpenGL}
\subtitle{DH2640-Graphics and Interaction Programming}
\institute[KTH]{KTH -- Stockholm, Sweden}
\author{Arnaud RAMEY}
\date{June 16., 2010}



\mode<presentation>
%W/o navigation bar: default, boxes, Bergen, Madrid, Pittsburgh, Rochester
% With a tree-like navigation bar: Antibes, JuanLesPins, Montpellier.
% With a TOC sidebar: Berkeley, PaloAlto, Goettingen, Marburg, Hannover
% With a mini frame navigation: Berlin, Ilmenau, Dresden, Darmstadt, Frankfurt, Singapore, Szeged
% With section and subsection titles: Copenhagen, Luebeck, Malmoe, Warsaw
\usetheme{shadow}
\usetheme{Pittsburgh} % bests : Pittsburgh , Goettingen, Hannover
\usecolortheme{seagull} % albatross, beetle, crane, dove, fly, seagull, lily
\usefonttheme{professionalfonts}  %default, professionalfonts, serif, structurebold, structureitalicserif, structuresmallcapsserif


\setbeamertemplate{items}[default]
%\setbeamertemplate{enumerate}[default]
\setbeamertemplate{sections/subsections in toc}[sections numbered]
%\setbeamertemplate{blocks}[rounded][shadow=true]
\setbeamertemplate{navigation symbols}{}

\useoutertheme{infolines}
\setbeamertemplate{headline}[default]          % removes toc above
%\setbeamertemplate{footline}[default]          % removes title under

\setlength\fboxsep{0 mm}
\newcommand{\putat}[3 ]{\begin{picture}(0,0)(0,0)\put(#1,#2){#3}\end{picture}}

\newcommand{\imageWithoutBox}[2] { % filename, height
    \includegraphics[height=#2 cm]{./images/#1.png}
}

\newcommand{\imageWithoutBoxAndExtension}[2] { % filename, height
    \includegraphics[height=#2 cm]{./images/#1}
}

\newcommand{\image}[2] { % filename, height
    \fbox {  \includegraphics[height=#2 cm]{./images/#1.png} }
}

\newcommand{\imageCentered}[2] { % filename, height
    \begin{center}
    \fbox {  \includegraphics[height=#2 cm]{./images/#1.png} }
    \end{center}
}


%%%%%%%%%%%%%%%%%%%%%%%%%%%%%%%%%%%%%%%%%%%%%%%%%%%%%%%%%
\begin{document}
\selectlanguage{english}

\AtBeginSection[] {
\begin{frame}<beamer>
        \frametitle{Contents}
        \tableofcontents[currentsection]
\end{frame}}

\newcommand{\maketitlepage} { {
    \setbeamertemplate{footline}[default]
    \setbeamertemplate{headline}[default]
    \frame{
        \titlepage
        \putat{-10}{20} {\imageWithoutBox {horse2} {3}}
        \putat{225}{20} {\imageWithoutBox {body2} {3}}
    }
    }
}

\maketitlepage

\begin{frame}<beamer>
    \frametitle{Contents}
    \tableofcontents[]
\end{frame}


%     \section{Purposes}
% \begin{frame}[fragile]
%     \frametitle{Purposes}
% \end{frame}

    \section{Skeleton animation}
\begin{frame}[fragile]
    \frametitle{Skeleton animation}
    \begin{center}
%     \image {bones} 4
    \image {skeleton} 4
    \image {bones2} 4
    \end{center}
\end{frame}

\begin{frame}[fragile]
    \frametitle{Representation of a joint}
    \vspace*{-1 cm}
    \begin{verbatim}
    class Joint {
        Joint father;
        Joint[] children;
        Point3 relative_position;
        Point3 relative_scale;
        Point4 relative_rotation;
    }
    \end{verbatim}
    \vspace*{-.8 cm}
    $\Rightarrow$ Compute the relative matrix of the joint (4x4)\\
    $\Rightarrow$ Compute the absolute matrix of the joint\\
    $absolute = father.absolute \times relative$ (4x4) \\
    \vspace*{.5 cm}
    ** Demo **
\end{frame}

    \section{Exporting data from Maya to OpenGL}
\begin{frame}[fragile]
    \frametitle{Importing data from Maya}
    Maya : a 3D modeller. \\
    Export objects according to the RTG format.\\
    C++ RTG parser, written by Gustav.
\end{frame}


    \section{Slicing the RTG mesh}
\begin{frame}[fragile]
    \frametitle{Slicing the RTG mesh}
    \textbf{Idea :} bind an RTG object to each joint\\
    %
    **Demo**\\
    %
    \textbf{Problem :} joints not perfect\\
    %
    \imageCentered {horse} 5
\end{frame}


    \section{Improvement : one-piece RTG + deformation}
\begin{frame}[fragile]
    \frametitle{Maya weight maps}
    \imageCentered{weights}{6}
\end{frame}

\begin{frame}[fragile]
    \frametitle{Improvement : one-piece RTG + deformation}
    Export the weight maps from Maya as images.
    \image{body}{3.5} \hspace*{-1cm}
    \image{uv}{3.5} \hspace*{-.5 cm}
    \image{spine_front}{3.5}
\end{frame}

\begin{frame}[fragile]
    \frametitle{How to deform the mesh ?}
    \begin{enumerate}
        \item \textbf{Reference poses} (= bind poses) for the mesh and the skeleton : compute the matrix\\
        $INVERSE \overset {DEF} = absolute^{-1}$ (constant 4x4)
        \item \textbf{At a random pose}, for each joint :
        $finale = absolute \times INVERSE$ (4x4)
        \item position of $p(x, y, z)$ :\\
        $$\tilde p = \frac
        {\underset {b \in bones} \sum weight(p, b) \times finale_b \times (x, y, z, 1) }
        {\underset {b \in bones} \sum weight(p, b)}
        $$
    \end{enumerate}
\end{frame}

\maketitlepage

\end{document}
